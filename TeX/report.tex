\documentclass[12pt,a4]{article}
\usepackage{amsmath}
\usepackage{amssymb}
\usepackage{amsfonts}
\usepackage{array}
\usepackage{graphicx}% use this package if an eps figure is included.
\usepackage{mathrsfs}
\usepackage{multirow}
\usepackage{siunitx}
\setlength\topmargin{-1.1in} \addtolength\textheight{2.1in}
\addtolength{\oddsidemargin}{-0.2in}
\addtolength{\evensidemargin}{-0.1in} \textwidth 5.8in
\newcounter{questioncounter}
\newcounter{equestioncounter}
\setlength\parskip{10pt} \setlength\parindent{0in}
\newcommand{\bea}{\begin{eqnarray*}}
\newcommand{\eea}{\end{eqnarray*}}
\newcommand{\beao}{\begin{eqnarray}}
\newcommand{\eeao}{\end{eqnarray}}
\newcommand{\no}{\noindent}

\begin{document}
\title{Physics-discovery of dynamical systems}


\author{Pranav Satheesh, Tim Capllice\
\small }
\maketitle


\begin{abstract}
A temperature wave propagates along a long thin bar of a metallic sample when subjected to periodic heating. In this way it is demonstrated that there is no wave nature in these improperly called thermal waves by showing that they do not transport energy and its propagation properties can be used to determine the thermal diffusivity of the material.
\end{abstract}

\section{Dynamical systems}

\section{Sparse Identification of Non-Linear Dynamics (SINDy)}


\subsection{Example: Lorenz system}

\subsection{Example: Rossler system}


\section{Koopman theory}

\section{Dynamic Mode Decomposition}



\section{Conclusion}
The purpose of this experiment was to measure the thermal diffusivity of the copper metal. Hence that purpose is achieved and the experimental value is close enough to the theoretical value \cite{4}. This experiment provides an opportunity to get acquainted with heat conduction in a way that is essentially different from that of classical experiments on stationary heat transmission. This experiment also allows one to learn thermal diffusivity measuring techniques in a simple and pedagogical way.

\begin{thebibliography}{99}

\bibitem{1} A. Mandelis, L. Nicolaides, Y. Chen, ``\emph{Structure and the reflectionless/refractionless nature of parabolic diffusion-wave fields}", Phys. Rev.Lett. \textbf{87}, 020801-1---020801-4 (2001).

\bibitem{2} A. Bodas, V. Gandia, E. Lopez-Baeza, ``\emph{An undergraduate experiment on the propagation of thermal waves}", Am. J. Phys. \textbf{66}, 528-1-533 (1998).

\bibitem{3} L. Verdini and A. Santucci, ``\emph{Propagation properties of thermal waves and
thermal diffusivity in metals}", Nuovo Cimento B \textbf{62}, 399-421 (1981).


\bibitem{4} M. S. Anwar, J. Alam, M. Wasif, R. Ullah, S. Shamim, W. Zia, ``\emph{Fourier analysis of thermal diffusive waves}", J. Phy \textbf{82}, 928 (2014).

\bibitem{5} George B. Arfken, Hans J. Weber, ``\emph{Mathematical methods for physicists}", 6th Ed, Chapter 14.

\bibitem{6} K. Etori, ``\emph{Remarks on the temperature propagation and the thermal diffusivity of a solid}", Jpn. J. Appl. Phys. \textbf{11}, 955-957 (1972).

%\bibitem{5} A. Mandelis, ``\emph{Diffusion waves and their uses}", Phys. Today \textbf{66}, 29-34 (2000).

\end{thebibliography}


\end{document} 